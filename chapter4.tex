\section{Conclusion}
\label{chap:Conclusion}

\subsection{PURE}

In the doctoral dissertation, we present the the approaches that identify the causal drugs/chemicals that have impact on gene expression profiles. On one hand, it can be used in pin-point the abundance harmful substances in the subject's system, and hence helps doctors to find a suitable treatment. On the other hand, these methods are extremely useful for researchers and doctors in repurpose drugs for alternative application, especially for new viruses or diseases such as COVID-19. These two applications are realized by two following studies.

In Chapter 2, we introduce a method, PURE, that tested two hypotheses that a CDT is responsible for the gene expression changes, which in turn causes the observed phenotype: the level of the CDT in the subject's system is higher than usual or the level of the CDT in the subject's system is lower than usual. The first hypothesis is a crucial ability for the correct identification of the presence of chemicals, drugs or toxicants in studied phenotypes.  With the second hypothesis, PURE can identify the CDT that can revert disease-induced gene expression changes. This capability is expected to be useful in any drug repurposing application. 
The proposed approach was validated using 16 gene expression data sets from 3 different species where the true CDTs that caused the phenotypes were known. We also compared PURE to 5 other methods including a commercial tool, IPA. PURE outperformed all other methods in terms of the rank of the true CDT and the number of false positives in the list of significant CDTs.

In Chapter 3, we used another approach to identify upstream CDTs coupled with a pathway analysis, to investigate an alternative treatment for severe symptoms related to hyper-inflammation of COVID-19 patients~\cite{DraghiciCOVID:2021}. The gene ontology analysis and pathway analysis methods on studied datasets have both confirmed that the cytokine storm is found in COVID-19 samples in this study. Applying the approach to identify upstream regulators would derive a list of proposed drugs that potentially reverse the DE genes and therefore, suppress the cytokine storms. We validated the results with another gene expression dataset. Subsequently, we will validate the proposed drug with an independent clinical trial.