\documentclass[Minh_PhD_thesis.tex]{subfiles}
\begin{document}

\section{Conclusion}

%In this work, we described the two main approaches that are normally conducted to study a given a gene expression profile: pathway analysis, and upstream analysis.
In this work, we described a study framework for drug repurposing to treat severe COVID-19 patients. We identified MP as a potential drugs that could improve the outcome of these patients. The results were validated by using additional datasets, and more importantly by a clinical trial in which the group of patients treated by MP has the thirty day all-case mortality rate significantly lower than patients in control group. 

In the early stage of COVID-19 pandemic, WHO published a guideline that was against the use of using steroid for treating COVID-19 patients. In the late 2020, WHO updated the guideline that recommend MP, the drug that we have found, for treating severe cases of COVID-19. 
%Later, the confirmation by WHO that corticosteroid can be used to treat patients with severe and critical COVID-19 again proves the legitimacy of our method.

During this process of identifying a potential drug, we encountered two major problems. First, we aimed to select a pathway analysis method to investigate the mechanisms and pathways affected in COVID-19 samples. However, the abundance of proposed methods presents a challenge. While several review papers exist on this topic, they are somewhat limited in scope. Secondly, an effective drug repurposing method has yet to be established.

%Given a gene expression profile, pathway analysis can provide the information of which pathways are impacted and the insight of the mechanism activated by the DE genes. 
To address the first problem, we provided a detailed discussion of 2 groups of pathway analysis methods: non-TB and TB methods. We presented 13 of the most popular tools in each groups and provided guidance for the researchers to choose a suitable method for their analysis. Not only analyzing the methods from theory aspect, we compared the performance some of the most widely used methods on 97 real data sets under three criteria: the ability to identify the truly impacted pathway (the target pathway), the area under the ROC curve using knock-out data sets, and the bias when the null hypothesis is true. This approach is objective and reproducible and can be used as a guidance for researchers to choose the most suitable methods for their analysis. For pathway analysis method developers, this process can be applied to evaluate their proposed method to the existing ones. 

%While pathway analysis methods can help understanding the impacted and active pathways, the upstream analysis can actual help us to either identify the cause of the phenotype, and/or identify potential CDTs that is supposed to suppresses the DE genes' expression. 

Regarding the second problem, we proposed an upstream analysis that not only can identify the cause of the phenotype, but also identify the potential CDTs that is supposed to suppresses the DE genes' expression. 
Our method infers the CDTs responsible for the changes in the gene expression profile, either because their level in the subject's system is higher (hypothesis 1) or lower (hypothesis 2) than normal. 
%This application is a crucial step to treat conditions related to external CDTs since it helps identifying the correct causal CDTs causing observed list of DE genes. 
On one hand,  hypothesis 1 helps identifying the cause of conditions related to external CDTs, and therefore is a crucial step for the treatment process. 
Our proposed method can be applied to time series analysis experiments in which the subjects' gene expression profiles are measured periodically after taking a known medicine. Applying our methods on these profiles would help identify the time point beyond which the effect of the drug ceases to affect the gene expression profiles in a significant way  (e.g. experiments 6-8 in Table \ref{Datasets}).
On the other hand, hypothesis 2 can be interpreted as a prediction of what is lacking in the system but also as a suggestion for a CDT that could reverse the expression changes induced by a disease phenotype, and therefore, is useful in the drug repurposing study. %Recently, a similar approach was applied to the discovery of a treatment for severe cases of COVID-19~\cite{DraghiciCOVID:2021}. 

For validation process, we would used 14 gene expression data sets with different known causal factors and different species for validating the performance of testing H1, and another two data sets for assessment of the ability of testing the H2. %The significance score would show whether our method is more robust than other classical and commercial approaches included in this study, in term of the rank of true CDT, and the number of false positives.

Subsequently, we applied these two approaches mentioned above to study the condition of the COVID patients as well as discover a potential treatments for these patients using the upstream analysis method. Using the pathway analysis method, we identified the cytokine-cytokine receptor interaction pathway that was highly impacted in these patients suggesting that the cytokine storm syndrome was developing in these patients. Applying the upstream analysis, we discover that methylprednisolone (MP) has the highest potential to reverse the expression of the DE genes in these patients' gene expression profile, and therefore can be used to treat patients with severe symptoms related to hyper-inflammation. 

%The drug was validated in an independent clinical trial and the result showed that MP can reduce the mortality rate in MP-treated group by 43\%.

%This study presents an approach for drug repurposing based on identifying drugs that could revert
%gene expression changes associated with most perturbed biological processes and pathways. 
Results from a clinical study undertaken in a cohort of 213 patients in a multi-center hospital system confirmed the efficacy of the \emph{in silico} prediction that indicated MP could improve outcomes in severe COVID-19. 
This prediction is also supported by the results of independent clinical studies with the same drug undertaken in Italy (173 patients) and Spain (85 patients). The drug repurposing approach described here, as well as the drugs identified, might be important for any future pandemic involving hyper-inflammation.


\end{document}
