\section{Problem Description}
\label{chap:ProblemDescription}

Everyday, we are surrounded by or/and exposed to many different chemicals, drugs, and toxicants in different shapes and forms, which could be natural origin, raw material, or produced by human activities. Some of them are harmful to our body, such as  carbon monoxide (CO) emitted by combustion engine or wildfires, whereas other are vital to our existence, such as vitamin or minerals, of course with a certain amount. Exposure to cigarette smoking, fumes, gases, or irritant chemicals are major risk for chronic obstructive pulmonary disease ~\cite{boschetto2006chronic} or numerous cancer sites such as lung, skin, liver, etc.~\cite{clapp2008environmental}. Longnecker \textit{et al.} discovered the association between DDT, a chemical used in insecticide, with preterm births, which is a major contributor to infant mortality~\cite{longnecker2001association}.
According to Prüss-Ustün \textit{et al.}, toxic exposure to chemicals were linked to 4.9 million deaths in 2004 (8.3\% of the total) \cite{pruss2011knowns}. Also in this article, the authors concluded that the most common toxic chemicals contributed to these deaths are indoor smoke from solid fuel use, air pollution, second-hand smoke, etc. Moreover, many other chemicals, e.g. used in pesticides, are known to have severed impact on human health.

We hypothesize that the diseases or phenotypes caused by expose or lack of specific chemical substances can be captured by the changes in gene expression profile resulting a list of differentially expressed genes.
This necessitates research aimed at inferring the causal factors, such as chemicals/drugs/toxicants (CDT), underlying high-throughput gene expression profiles.

On one hand, the identification of the chemical, drug, or toxicant (CDT) responsible for perturbing patients' gene expression levels facilitates the discovery of appropriate treatment modalities for affected individuals (causal analysis).
On the other hand, identifying the chemicals or drugs lacking in the system is extremely useful in finding (alternative) treatments for studied conditions (e.g. drug repurposing), since consuming these drugs can potentially flip the signs of the expressions of the DE genes and hence suppress the phenotype (drug repositioning application).